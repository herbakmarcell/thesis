%
% Szakdolgozatminta az Eszterházy Károly Katolikus Egyetem
% matematika illetve informatika szakos hallgatóinak.
%

\documentclass[
% opciók nélkül: egyoldalas nyomtatás, elektronikus verzió
% twoside,     % kétoldalas nyomtatás
% tocnopagenum,% oldalszámozás a tartalomjegyzék után kezdődik
]{thesis-ekf}
\usepackage[T1]{fontenc}
\PassOptionsToPackage{defaults=hu-min}{magyar.ldf}
\usepackage[magyar]{babel}
\usepackage{mathtools,amssymb,amsthm,pdfpages}
\footnotestyle{rule=fourth}

\newtheorem{tetel}{Tétel}[chapter]
\theoremstyle{definition}
\newtheorem{definicio}[tetel]{Definíció}
\theoremstyle{remark}
\newtheorem{megjegyzes}[tetel]{Megjegyzés}

\begin{document}

\institute{Matematikai és Informatikai Intézet}
\title{A szakdolgozat címe}
\author{Herbák Marcell\\Programtervező Informatikus BSc}
\supervisor{Dr. Kovásznai Gergely\\Egyetemi docens}
\city{Eger}
\date{2025}
\maketitle

\tableofcontents

\chapter*{Bevezetés}
\addcontentsline{toc}{chapter}{Bevezetés}


\chapter{Mesterséges intelligencia}

\section{Története}

\chapter{Technológiák}

\section{Unity}

\chapter{Megvalósítás}

\chapter{Tesztelés}

\chapter*{Összegzés}
\addcontentsline{toc}{chapter}{Összegzés}


\begin{thebibliography}{2}
\addcontentsline{toc}{chapter}{\bibname}
\end{thebibliography}

% Aláírt, szkennelt nyilatkozat beillesztése a szakdolgozat végére
\includepdf{nyilatkozat/nyilatkozat.pdf}
\end{document}